\documentclass{standalone}
\usepackage{tikz}
\usepackage{amsmath}
\begin{document}

% load matrix from tikz
\usetikzlibrary{matrix}

% write model within tikzpicture
\begin{tikzpicture}
% start matrix
% arg list
% 1. specifies matrix of "math" nodes so that you can type math for nodes
% 2. space between columns
% 3. space between rows
% (mat) is our name for the matrix and we use the name to reference nodes
\matrix[matrix of math nodes, column sep=30 pt, row sep=30 pt] (mat)
{
    % & separates nodes and \\ begins new lines
    & \mu, \tau^2 & \\ 
    \theta_1 & \ldots & \theta_n \\
    y_1 & \ldots & y_n \\
    & \sigma^2 & \\
};

% loop over columns 1 and 3 to draw arrows more easily
% \column is just the name of the variable to hold the values 1 and 3
\foreach \column in {1, 3}
{
    % -> and <- specify direction of arrow
    % >=latex specifies latex style for arrow head
    % after \draw command, specify which nodes you draw from and to
    % first number represents row of node, second is column
    \draw[->,>=latex] (mat-1-2) -- (mat-2-\column);
    \draw[->,>=latex] (mat-2-\column) -- (mat-3-\column);
    \draw[<-,>=latex] (mat-3-\column) -- (mat-4-2);
}
% now we annotate some of the rows
%\node specifies whether the anchoring node is to the east or west
% xshift specifies distance away from which node
\node[anchor=east] at ([xshift =-20pt]mat-2-1) {$\theta_i \sim \text{N}(\mu, \tau^2)$};
\node[anchor=east] at ([xshift =-20pt]mat-3-1) {$y_i \sim \text{N}(\theta_i, \sigma^2)$};
\end{tikzpicture}
\end{standalone}
